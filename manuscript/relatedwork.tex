\section{Related Work}

This work concerns to the research topic of spatial data visual analytics. Spatio-temporal data has three components, i.e., spatial, temporal and thematic~\cite{andrienko2013visual}. Lots of related work focus on the spatio-temporal analysis in movement data, in the absence of thematic information. The presense of geo-tagged social media data flourishes the exploration of thematic information along with movement. By analyzing the texts, those work explain what drive the moving activities or what is resulted from. This work continues the thematic research in movement data, but focus more on individual charecteristics. Landed by a census experiment, this work is able to analyze the profile information directly, to get rid of indirect data inferred from social media data as other related work do.

\textbf{Movement Data Analysis} 
 
 With the development of location-acquistion techniques, massive spatial trajectories are collected, to keep track of the trajectories of various moving objects. Many techniques have been proposed to process, mining trajectory~\cite{Zheng2015_trajectory}. In the field of visualization and visual analytic, spatial visualizations are specifically designed for the time, locations, spatial-temporal information and other properties in the traffic data~\cite{chen2015survey}. A large number of visual analytics tools and applications cover situation-aware exploration, pattern discovery and traffic situation monitoring. Wang et al~\cite{wang2013visual} extract the traffic jam propagation graph extraction to reveal underlying data patterns. Guo et al.~\cite{guo2011tripvista} and Zeng et al.~\cite{zeng2013visualizing} et al. construct geographical regions and visually aggregate the in-between movements as flows. To discover the route travel patterns, Lu et al.~\cite{lu2015trajrank} propose TrajRank to explore the route travel behaviour based on ranking. For multiple routes, Liu et al.~\cite{liu2011_routediversity} study the route diversity between locations and Lu et al.~\cite{Lu2017_multipleroute} explore the route choice behaviour among multiple routes.


\textbf{Geo-tagged Social Media Data Analysis}

As the prevence of social medial serivces, social media data with geo-tags are collected to track people's movements in daily lifes. As an analogue of remote sensing data in social science research, the geospatial big data has been proposed as social sensing ~\cite{liu2015social}. Hence, analyzing movement information along with rich text has become a hot research area in recent years. Those works infer thematic information from the semantic texts. Cao et al.~\cite{cao2012whisper} propose \textit{Whisper} for tracing the pathways of tweets on a spatial hierarchical layouts, to investigate how information flow among multiple places. Krueger et al.~\cite{krueger2014visual} used GPS and location-based service data to support the analysis of movement behaviors. Chen et al.~\cite{chen2016interactive} present a visual analytics system to support the exploraiton of sparcely sampled trajectory from social media. Some other researches tend to infer real information, such as names, gender, etc~\cite{peddinti2014internet}, to break down the demographic characteristics of social media users. Luo et al.~\cite{luo2016explore} derive race, gender and age as three demographics dimensions to analyze its impact on the urban human mobility patterns. Since real information is not enforced in social media, wrong infering is inevitable. Longley et al.~\cite{Longley2015}~\cite{Paul2016_twitter} identifies and assess the biases inherent in social media usage in social research and evaluate the deployment of social media data in research applications.  







