\section{Discussion}

\textbf{Compatibility}

Compared to other static visual cuing, animation effect has better compatibility by not importing other markers. 

\textbf{Mapping}

Kinetic effects is driven by data with what kind of mapping, linear, non-linear? relationship to perceptual driven, tuned by perception. 


\textbf{Visual Tired/ Distracting the reading of other visual channel}

 Does it disturb the reading of original static charts?
 
 What we show in this work is to add kinetic effect all in, the extreme case of visual information enhancement. If applying to certain part of visual objects, it would for sure significantly reduce the visual distraction and facilitate the attention attraction.
  
\textbf{Interactive Enhancement / Adaptive Dynamic Control}

For sure the kinetic effects can be used in the interactive visual analytics system, as the cuing techniques after interactions, e.g., annotation, filtering, query, etc. For most of visual analytics, existing visual cuing is static and because of buried in rich visual encoding, it is usually hard to stand out. Kinetic effects can be considered as efficient attention attraction encoding. 

Also, it also allows the users' manual control on playing/stopping kinetic effect, to call the enhancement of information on demand

Adaptive dynamic enhancement, e.g., sleep the kinetic effects ?

