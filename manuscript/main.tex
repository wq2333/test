
\documentclass[journal]{vgtc}                % final (journal style)
%\documentclass[review,journal]{vgtc}         % review (journal style)
%\documentclass[widereview]{vgtc}             % wide-spaced review
%\documentclass[preprint,journal]{vgtc}       % preprint (journal style)

%% Uncomment one of the lines above depending on where your paper is
%% in the conference process. ``review'' and ``widereview'' are for review
%% submission, ``preprint'' is for pre-publication, and the final version
%% doesn't use a specific qualifier.

%% Please use one of the ``review'' options in combination with the
%% assigned online id (see below) ONLY if your paper uses a double blind
%% review process. Some conferences, like IEEE Vis and InfoVis, have NOT
%% in the past.

%% Please note that the use of figures other than the optional teaser is not permitted on the first page
%% of the journal version.  Figures should begin on the second page and be
%% in CMYK or Grey scale format, otherwise, colour shifting may occur
%% during the printing process.  Papers submitted with figures other than the optional teaser on the
%% first page will be refused. Also, the teaser figure should only have the
%% width of the abstract as the template enforces it.

%% These few lines make a distinction between latex and pdflatex calls and they
%% bring in essential packages for graphics and font handling.
%% Note that due to the \DeclareGraphicsExtensions{} call it is no longer necessary
%% to provide the the path and extension of a graphics file:
%% \includegraphics{diamondrule} is completely sufficient.
%%
\ifpdf%                                % if we use pdflatex
  \pdfoutput=1\relax                   % create PDFs from pdfLaTeX
  \pdfcompresslevel=9                  % PDF Compression
  \pdfoptionpdfminorversion=7          % create PDF 1.7
  \ExecuteOptions{pdftex}
  \usepackage{graphicx}                % allow us to embed graphics files
  \DeclareGraphicsExtensions{.pdf,.png,.jpg,.jpeg} % for pdflatex we expect .pdf, .png, or .jpg files
\else%                                 % else we use pure latex
  \ExecuteOptions{dvips}
  \usepackage{graphicx}                % allow us to embed graphics files
  \DeclareGraphicsExtensions{.eps}     % for pure latex we expect eps files
\fi%

%% it is recomended to use ``\autoref{sec:bla}'' instead of ``Fig.~\ref{sec:bla}''
\graphicspath{{figures/}{pictures/}{images/}{./}} % where to search for the images

\usepackage{microtype}                 % use micro-typography (slightly more compact, better to read)
\PassOptionsToPackage{warn}{textcomp}  % to address font issues with \textrightarrow
\usepackage{textcomp}                  % use better special symbols
\usepackage{mathptmx}                  % use matching math font
\usepackage{times}                     % we use Times as the main font
\renewcommand*\ttdefault{txtt}         % a nicer typewriter font
\usepackage{cite}                      % needed to automatically sort the references
\usepackage{tabu}                      % only used for the table example
\usepackage{booktabs}                  % only used for the table example
%% We encourage the use of mathptmx for consistent usage of times font
%% throughout the proceedings. However, if you encounter conflicts
%% with other math-related packages, you may want to disable it.
\usepackage{wrapfig}
\usepackage{subfig}
\usepackage{caption}

\usepackage{hyperref}
\usepackage[usenames, dvipsnames]{color}
%% In preprint mode you may define your own headline.
%\preprinttext{To appear in IEEE Transactions on Visualization and Computer Graphics.}

%% If you are submitting a paper to a conference for review with a double
%% blind reviewing process, please replace the value ``0'' below with your
%% OnlineID. Otherwise, you may safely leave it at ``0''.
\onlineid{0}

%% declare the category of your paper, only shown in review mode
\vgtccategory{Research}
%% please declare the paper type of your paper to help reviewers, only shown in review mode
%% choices:
%% * algorithm/technique
%% * application/design study
%% * evaluation
%% * system
%% * theory/model
\vgtcpapertype{algorithm/technique}

\newcommand\addcomment[1]{\textcolor{red}{#1}}

\newcommand{\specialcell}[2][l]{%
\begin{tabular}[#1]{@{}l@{}}#2\end{tabular}}

%% Paper title.
%\title{Data-Driven Animation Effect Enhancement to Activate Static Visual Charts}
\title{City Folding: Visual Exploration of Urban Dynamics based on Deomgraphic Data Collected via Social Media}

% City Folding: A Map Distortion Method Driven by Social Deomgraphics Data

% Visual Exploration of Urban Demographic Visual Exploration of Urban Data


%% This is how authors are specified in the journal style

%% indicate IEEE Member or Student Member in form indicated below
%\author{Qi Wang, Min Lu, Yang Yue, ...}
%\authorfooter{
%% insert punctuation at end of each item
%\item
%Qi Wang, Min Lu and Hui Huang are with Shenzhen University. E-mail: \{minlu, huihuang\}@szu.edu.cn
%\item
% Noa Fish is with Tel Aviv University, E-mail: {noafish}@post.tau.ac.il
%\item
%Dainel Cohen-Or is with Tel Aviv University and Shenzhen University, E-mail: {dcor}@tau.ac.il
%\item 
%To whom correspondence should be addressed,
%email: {huihuang}@szu.edu.cn
%}

%other entries to be set up for journal
%\shortauthortitle{Lu \MakeLowercase{\textit{et al.}}: Enhancing Static Visual Charts}
%\shortauthortitle{Firstauthor \MakeLowercase{\textit{et al.}}: Paper Title}

%% Abstract section.
\abstract{A city is shaped not only by its natural geographic landscape but also by social accessiblity to various citizens. In this work, via a social wide census investigation via the social media platform, citizens with various demographics are reached. We propose a map distortion method to provide an intuitive perception of the accessiblity, i.e., the kind of city folding phenomenon, in the context of various levels of social attributes (e.g., income, education, etc). Accompanied with temporal and spatial multiple views, the method is capabile to reveal the detail of levels.

\iffalse

For centuries, knowledge sharing and exchange of ideas have relied upon the usage of visual charts and diagrams designed to be viewed in hard-copy form.
To emphasize and draw attention to certain aspects of the data, static cues like color and geometric shapes were utilized with great success.
Nowadays, digital displays are ubiquitous in the visualization of any form of data, lifting the confines of static presentations.
In this work, we propose taking a natural step forward, and incorporate data-driven dynamic enhancements into otherwise static visualization schemes, for the purpose of element emphasis and attention guidance. 
Given a chart or a diagram, and their underlying data, we perform a simple analysis to determine elements and attributes of importance. According to their characteristics, a suitable dynamic effect is applied directly onto the displayed data, creating an illusion of a visualization brought to life.
We experiment with three versatile effects, namely, \textit{marching ants}, \textit{geometry deformation} and \textit{blinking}, and provide practical details regarding their mode of operation and extent of interaction with existing visual channels.
We examine the impact and effectiveness of our enhancements via two user studies designed to assess personal preference as well as gauge the influence of visual dynamic effects on human perception in terms of fast, yet accurate visual understanding. 

Complementary to conventional visual channels such as color, shape, in this work, we propose to enhance the static charts with activating visual channel, i.e., animation effect, to bring life to visualization. The usage of animation effect enhancement is instantiated in different data types, including graph, tree, spatial, temporal, etc. A user study reports its effectiveness. 
\fi
%
} % end of abstract

%% Keywords that describe your work. Will show as 'Index Terms' in journal
%% please capitalize first letter and insert punctuation after last keyword
\keywords{Demographics, Spatial-temporal Visualization, Map Distortion}

%% ACM Computing Classification System (CCS). 
%% See <http://www.acm.org/class/1998/> for details.
%% The ``\CCScat'' command takes four arguments.

\CCScatlist{ % not used in journal version
 \CCScat{K.6.1}{Management of Computing and Information Systems}%
{Project and People Management}{Life Cycle};
 \CCScat{K.7.m}{The Computing Profession}{Miscellaneous}{Ethics}
}

%% Uncomment below to include a teaser figure.
% \teaser{
%   \centering
%   \includegraphics[width=\linewidth]{CypressView}
%   \caption{In the Clouds: Vancouver from Cypress Mountain. Note that the teaser may not be wider than the abstract block.}
% 	\label{fig:teaser}
% }

%% Uncomment below to disable the manuscript note
%\renewcommand{\manuscriptnotetxt}{}

%% Copyright space is enabled by default as required by guidelines.
%% It is disabled by the 'review' option or via the following command:
% \nocopyrightspace

\vgtcinsertpkg

%%%%%%%%%%%%%%%%%%%%%%%%%%%%%%%%%%%%%%%%%%%%%%%%%%%%%%%%%%%%%%%%
%%%%%%%%%%%%%%%%%%%%%% START OF THE PAPER %%%%%%%%%%%%%%%%%%%%%%
%%%%%%%%%%%%%%%%%%%%%%%%%%%%%%%%%%%%%%%%%%%%%%%%%%%%%%%%%%%%%%%%%

\begin{document}

%% The ``\maketitle'' command must be the first command after the
%% ``\begin{document}'' command. It prepares and prints the title block.

%% the only exception to this rule is the \firstsection command
\firstsection{Introduction}

\maketitle

%% \section{Introduction} %for journal use above \firstsection{..} instead
%Without interference with other visual encoding, animation effect is proved as a light-weight extra and preattensive cue. 

Collecting data via a widely used social medial platform, users (what the right word here?) with various profiles are reached and assembled as rich (and fair?) demographic data. 



\section{Related Work}

The related work is discussed around the related methods in map distortion, and the work analyzing facility accessbility....

\paragraph{Map Distrotion}

\paragraph{Facility Accessibility Analysis}






\section{Online Census}

The advent of mobile sensing techniques and social media applications makes it possible to collect spatial data from the social media source. Complementary to the conventional census, it brings the benefit of larger sampling frequency and a broader range in terms of space and time. It is possible to reach a wide range of individuals and collect the movement in human inactive time, such as the mid-night.


\begin{figure}[htb!]
 \centering % avoid the use of \begin{center}...\end{center} and use \centering instead (more compact)
 \includegraphics[width=\columnwidth]{pictures/survey_app}
 \caption{Census Interface: (a) personal characteristics collecting page; (b) trips collecting page; (c) credit system page}
 \label{fig:app}
\end{figure}

In this work, we perform the census survey in Shenzhen, which is one of the most modern metropolia in China. The experiment is deployed on Wechat, a widely used social media application. Figure~\ref{fig:app} shows the data collecting interfaces. Each individual hands in his or her personal characteristics. For privacy issue, all detailed personal information are desensitized to categorical levels (Figure~\ref{fig:app}(a)). 


\textbf{Individual Characteristics} Figure~\ref{fig:data_over} lists the \textit{eight domains}, including social, economic and demographic aspects, to give a generalized description of the individual characteristics. The profile will serve as the ingredients for the analysis of mobility patterns over diverse individuals. 


\begin{figure}[htb!]
 \centering % avoid the use of \begin{center}...\end{center} and use \centering instead (more compact)
 \includegraphics[width=\columnwidth]{pictures/data_over}
 \caption{Profile of Individual: eight individual characteristics enrich the analysis of mobility patterns}
 \label{fig:data_over}
\end{figure}

\textbf{Traveling Trips} Besides to those individual characteristics, each individual can upload dynamic traveling trips (Figure~\ref{fig:app}(b)). Each trip requires the information of \textit{start/end location}, \textit{start/end time}, \textit{traveling purpose}. To encourage the trip uploading, a credit system retains the contribution of individuals on trips and rewards the volunteers with the top credits (Figure~\ref{fig:app}(c)). 


\subsection{Basic Statistics of Data}

Over the releasing time period from \m{2015-11 to 2016-01}, 21435 individuals (48\% females and 52\% males) were reached and \m{229155} trips are collected. Each volunteer contributes \m{11} trips average. 

Our case-study data is confined to a small proportion of Wechat users who opt to contribute their information and trips.
Considering the caveat that self-selecting individuals are most unlikely to represent any clearly defined population~\cite{Longley2015}, we performed a series of preliminary statistics to check whether it is rich enough to represent a wide range of the population in the city. 

Figure~\ref{fig:data_age_edu} gives the distribution of age and education over the population. It shows that samples cover a wide range of ages, dominating between 18 to 45. There is also a few records pertaining to individuals below the age of 18 or above 70. The distribution follows the fact that Shenzhen is a city where the majority is young people. According to the 2015 Annual Census Statistics report\footnote{http://www.sztj.gov.cn/xxgk/tjsj/pcgb/201606/t20160614\_3697000.htm}, people aging 15-64 occupy 83.23\% and the median age is 31.5. Figure~\ref{fig:data_age_edu} gives the distribution of education levels, ranging from low to high. The technical college and university dominate the samples at the 61\% occupancy rate. 

% In the report (looking for some report), the penetration of mobile device is \m{XXX}, almost every XX people got a Mobile Phone in the urban. 

\begin{figure}[htb!]
 \centering % avoid the use of \begin{center}...\end{center} and use \centering instead (more compact)
 \includegraphics[width=\columnwidth]{pictures/data1}
 \caption{Age and Education Distribution: (a) age; (b) education}
 \label{fig:data_age_edu}
\end{figure}

Figure~\ref{fig:data_job_inc}(a) shows the job types of sampled individuals, who are servants, workers, officers, businessmen and so on. The covering of jobs is pretty wide. Figure~\ref{fig:data_job_inc}(b) gives the radar diagram of the annual pay. The majority get paid below 200 000. Individuals with higher salary are also reached in our census.

\begin{figure}[htb!]
 \centering % avoid the use of \begin{center}...\end{center} and use \centering instead (more compact)
 \includegraphics[width=\columnwidth]{pictures/data2}
 \caption{Job and Income Distribution: (a) job; (b) income}
 \label{fig:data_job_inc}
\end{figure}

Figure~\ref{fig:data_geometry} shows the basic statistics related to trips. In Figure~\ref{fig:data_geometry}(a), the active traveling time (here, we take the start time as the representative of active time) follows the common knowledge of urban life. There are obvious morning and late afternoon peaks. Figure~\ref{fig:data_geometry}(b) gives the counting of different traveling purposes. 95\% trips are tagged with clear raveling purpose in the data. 33\% are going home and 37\% are going to work. Besides this kind of routine traveling, there are also substantial trips such as going shopping, going the hospital, etc. Figure~\ref{fig:data_geometry}(c) shows the spatial distribution of the origins and destinations. It is found that more dots are located in Futian and Nanshan districts, the city's heart than the surrounding areas. This is consistent to Batty's exposition of the focus of city networks and interaction patterns~\cite{batty2013new}. 

% Trips are uploaded in a wide range of purpose in addition to home, work, but also includes XX entertainment, visiting, etc. The diverse purposes make it possible to characterize the flows across the city between different functional places. 

\begin{figure}[htb!]
 \centering % avoid the use of \begin{center}...\end{center} and use \centering instead (more compact)
 \includegraphics[width=\columnwidth]{pictures/data3}
 \caption{Statistics of Trips: (a) start time of trips; (b) purpose of trips; (c) distribution of origins/destinations }
 \label{fig:data_geometry}
\end{figure}

Because there is always inevitable bias inherent in fully representing the ground truth of the population, the preliminary statistical analysis shows positive sign of a relatively even sampling of the population. 
% As a note, although it is a subset of all people with 

\section{Idea}

Classification method, flow clustering method, Semantic representation. 

The key of our work is to align the analysis of urban dynamics with the available of individual characteristics. 

Problem to solve: 
\begin{itemize}
\item \textit{Groups of people}: multiple attributes, semantic understanding, loose attribute boundary
\item \textit{Urban Dynamics of People Group}: where those people go and how they access the ... in a city. the patterns of behaviour classified by different social characteristics.  
\end{itemize}

\section{City-folding Visual Analytics}

Semantic understanding of social characteristics.

\subsection{Data-driven Visual Profile}

to abstract to concrete and semantic understanding, to help users clues to do the classification. 

Face... Data-driven inforgraphics~\cite{}. 

Figure~\ref{fig:design_profile} shows the legend for the user profile. Those design dimensions are driven by data. an intuitive perception of the social characteristics. 

Considering the driven machinsm in two attributes: 
\begin{itemize}
\item \textit{Numerical}
\item \textit{Categorical}
\end{itemize}

\begin{figure}[htb!]
 \centering % avoid the use of \begin{center}...\end{center} and use \centering instead (more compact)
 \includegraphics[width=\columnwidth]{pictures/design_profile}
 \caption{Design Profile}
 \label{fig:design_profile}
\end{figure}

Various profiles are generated. Figure~\ref{fig:div_profile} shows some results.

\begin{figure}[htb!]
 \centering % avoid the use of \begin{center}...\end{center} and use \centering instead (more compact)
 \includegraphics[width=\columnwidth]{pictures/design_div}
 \caption{Diverse Profile}
 \label{fig:div_profile}
\end{figure}

\subsection{Interactive Group Classification}

Figure~\ref{fig:mds} is the MDS project of diversse users. Contour detected. Classification Updated. Automatic ensembling of diverse profiles. 

Groups from demographics

\begin{figure}[htb!]
 \centering % avoid the use of \begin{center}...\end{center} and use \centering instead (more compact)
 \includegraphics[width=\columnwidth]{pictures/mds}
 \caption{Interactive Labelling and Classification}
 \label{fig:mds}
\end{figure}

\subsection{Spatial Visualization}

The space visualization is embedded in 2.5D space. TAZ is used as the .... The height of space is put as the occurrence of visiting. For each group of people, we adjust the DB-Scan Algorithm in the context of TAZ. 
\section{Experts Feedback}

\lmc{Add a section to introduce the domain experts' feedback}


\lm{We interview XX domain experts from the GIS field. XX of them are with ... The procedure went as following. We first introduce them the system an...}

\section{Case Study}

\subsection{OVERVIEW Case 1: City Folding}

compare the distorted maps from various groups

\subsection{DETAIL Case 2: Look into Detail}

check the detail information of a group
% \section{Discussion}

\textbf{Compatibility}

Compared to other static visual cuing, animation effect has better compatibility by not importing other markers. 

\textbf{Mapping}

Kinetic effects is driven by data with what kind of mapping, linear, non-linear? relationship to perceptual driven, tuned by perception. 


\textbf{Visual Tired/ Distracting the reading of other visual channel}

 Does it disturb the reading of original static charts?
 
 What we show in this work is to add kinetic effect all in, the extreme case of visual information enhancement. If applying to certain part of visual objects, it would for sure significantly reduce the visual distraction and facilitate the attention attraction.
  
\textbf{Interactive Enhancement / Adaptive Dynamic Control}

For sure the kinetic effects can be used in the interactive visual analytics system, as the cuing techniques after interactions, e.g., annotation, filtering, query, etc. For most of visual analytics, existing visual cuing is static and because of buried in rich visual encoding, it is usually hard to stand out. Kinetic effects can be considered as efficient attention attraction encoding. 

Also, it also allows the users' manual control on playing/stopping kinetic effect, to call the enhancement of information on demand

Adaptive dynamic enhancement, e.g., sleep the kinetic effects ?



\section{Conclusion}

In this paper, we presented the map distrotion, whis driven by the accessibility of demographics. Our experimental findings indicate that ..., that aids in ...

Our current ...


 


% In this work, we propose to enhance static charts with kinetic effects, which serve as an plausible and orthogonal extension of conventional static visual cuing. Three kinetic effects (marching ants, geometry deformation and blinking) are introduced to demonstrate the idea from dynamic effects of motion, deformation and on-off repeating patterns. Kinetic effects encode various messages by design dimensions via the data driven mechanism. ...  kinetic effects brings the unsaid or less-notified information and bring the static infographics and charts to life. A controlled user study ...%Aligned with daily life experience , charts  enhanced by Kinetic Effect provides 
% which maintains the richness of expressibility . 


%% if specified like this the section will be committed in review mode
% \acknowledgments{
% The authors wish to thank A, B, and C. This work was supported in part by
% a grant from XYZ (\# 12345-67890).}

%\bibliographystyle{abbrv}
\bibliographystyle{abbrv-doi}
%\bibliographystyle{abbrv-doi-narrow}
%\bibliographystyle{abbrv-doi-hyperref}
%\bibliographystyle{abbrv-doi-hyperref-narrow}

\bibliography{main}
\end{document}

