\section{Overview}

\lmc{Add a paragraph to introduce the motivation of this work, where we derive the tasks via the interview with domain exports.}

The design of the system is driven by tasks, to align the analysis of mobility patterns with individual characteristics. Before diving into the design of the system, we introduce the couple of specifical tasks the system intended for. 

\begin{itemize}
\item \textit{Task 1: Identify the group with specific individual characteristics}: to get groups of people with common or close attributes, to explore the correlation among individual characteristics.
\item \textit{Task 2: Explore the mobility patterns of a group}: to explore where, how and why an interested group go in the city. 
\item \textit{Task 3: Compare mobility patterns among multiple groups}: to know the similarities and differences between different groups in their mobility patterns, to investigate the relationship between movement and individual characteristics.
\end{itemize}

\subsection{Design Considerations}

With these three tasks, we derive following design considerations:

\begin{itemize}
\item \textit{Intuitive perception of an individual as an organic complex (C1)}: the system should make use of users' daily life experience in knowing people to provide intuitive visualization, instead of the lifeless representation by number. The visual design needs to help end-users to pick desirable ones from the mass. 
\item \textit{Good overview of the multivariate individuals (C2)}: following the visual analytics manta by Shneiderman~\cite{RN459}, it is very important to provide a good overview of all the individuals then the users know where to explore. 
\item \textit{Effective multivariable cross-filter for individual characteristics (C3)}: there are eight domains to describe an individual. The system is supposed to provide a straight-forward way for easy filtering by the eight criteria. 
\item \textit{Compact visualization of mobility patterns in the constraint of spatial space (C4)}: the analysis of mobility patterns not only includes the conventional spatial and temporal dimensions, but also other abstract dimensions, e.g., travel purpose, visiting frequency, etc. The system should handle a compact layout to support the easy correlation between spatial and abstract information. 
\item \textit{Flexible interactions to explore the mobility patterns either within one group or between groups (C5)}: to support the comparison among groups, \textit{Task 3}, the system should maintain flexible interactions which allow the end-users to explore freely.
\end{itemize}

% Classification method, flow clustering method, Semantic representation. 

With those design considerations, we develop a visual analytics system as Figure~\ref{fig:teaser} shows. It composes of individual panel (the left part) for individuals (\textit{Task 1}) and spatial panel (the right part) for the mobility pattern (\textit{Task 2 and 3}). In the individual panel, interactive infographics, t-SNE visualization, and data-driven profiles support users to narrow the scope down to groups of individuals with interested characteristics. With the chosen group of individuals, its moving related information is visualized and explored in a 2.5D spatio-temporal view.


